% Glossary of typical IVOA terms.




\newcommand{\glossitem}[2]
{

\item \color{VOnum}{{#1}} :  \color{black}{{#2}}
}



\glossitem{ADQL}{(Astronomical Data Query Language) is the language used by the International Virtual Observatory Alliance (IVOA) to represent astronomy queries posted to VO services. ADQL is based on the Structured Query Language (SQL), especially on SQL 92. The VO has a number of tabular data sets and many of them are stored in relational databases, making SQL a convenient access means. A subset of the SQL grammar has been extended to support queries that are specific to astronomy.}
\glossitem{agent}{Software that acts or works on behalf of a user.}
\glossitem{AJAX}{(Asynchronous Javascript + XML) A framework for adding dynamic interactions within web pages.}
\glossitem{Aladin}{An interactive tool that allows the user to visualize digitized images and catalogs from many sources.}
\glossitem{ant}{A Java -based software build tool, similar to Unix ""make"".}
\glossitem{API}{(Application Programming Interface) The documentation of the interface to a software library or tool.}
\glossitem{applet}{A small program that runs in a larger client context, often Java programs embedded in Web pages.}
\glossitem{architecture}{The overarching design of a computer, network, or software system.}
\glossitem{array}{A data structure for software elements where each element has a unique identifying index number.}
\glossitem{ASCII}{(American Standard Code for Information Interchange) Formally, an encoding of common alphanumeric symbols. Often used to mean a human readable representation with no ‘special’ characters or formatting.}
\glossitem{ASP}{(Active Server Pages) A technology that enables dynamic web pages using server -side scripting.}
\glossitem{asynchronous services}{Web services where there may be a delay between the request for a resource (or service) and the response where the client is not expected to wait for the server.}
\glossitem{attributes}{1. In XML, characteristics of an element that are specified within the '<', '>', brackets after the name of the element, e.g. . 2. In database management systems, the term attribute is sometimes used as a synonym for field (i.e. column).}
\glossitem{bindings}{Associations between defined web interfaces and the services that provide them.}
\glossitem{authentication}{In computer security, verification of the identity of a user and/or the user's eligibility to access a service.}
\glossitem{C\#}{An object-oriented programming language from Microsoft that is based on C++ with elements from Visual Basic and Java.}
\glossitem{C++}{An object-oriented version of the C programming language.}
\glossitem{callbacks}{Routines that are automatically invoked when some event occurs or situation is encountered.}
\glossitem{Carnivore}{Open source registry developed at Caltech, supporting publishing, searching, and harvesting. Primarily intended for use by data providers who want to set up their own registry.}
\glossitem{certificate}{An electronic document that verifies the owner of a public key, issued by a certificate authority.}
\glossitem{CGI}{(Common Gateway Interface) A protocol that defines how data is passed to server applications using HTTP.}
\glossitem{client}{A computer program or terminal that requests information or services from another computer (a server) on the network.}
\glossitem{code stubs}{Code, usually generated by software tools, which defines the interfaces to some component but typically does not include any implementation of its functionality.}
\glossitem{cone search}{A VO protocol that requests information near a specified location in the sky.}
\glossitem{container}{An element that acts as a parent and contains child elements.}
\glossitem{CORBA}{(Common Object Request Broker Architecture) An open, vendor-independent architecture and infrastructure that computer applications use to work together over networks.}
\glossitem{crossmatch}{Find objects from two or more datasets that are near each other in the sky.}
\glossitem{cyberinfrastructure}{A research environment in which advanced computational services are available to researchers through high-performance networks.}
\glossitem{daemon}{A program or process that runs in the background unattended and may be invoked by another process to perform its function.}
\glossitem{DAL}{(Data Access Layer) The VO protocols that define how VO applications access data resources.}
\glossitem{data model}{A formal description of how data may be structured and used.}
\glossitem{database management system}{A collection of programs that enables storage, modification, and information extraction from a database. Also see RDBMS.}
\glossitem{DataScope}{A web-based VO tool that finds information from many VO sources near a specified point in the sky.}
\glossitem{distributed database}{A database where the underlying data is stored on multiple servers.}
\glossitem{distributed computing}{Spreading the workload for processing tasks over multiple machines.}
\glossitem{DCOM}{(Distributed Common Object Model) A protocol that allows communication and manipulation of objects over a network connection.}
\glossitem{DHCP}{(Dynamic Host Configuration Protocol) A protocol that automatically manages IP addresses for a set of nodes in a network.}
\glossitem{DOM}{(Document Object Model) A W3C standard in which a structured document such as an XML file is viewed as a tree of elements.}
\glossitem{.NET}{Microsoft's language-independent application development platform for creating web applications and web services.}
\glossitem{element}{1. A single item in an array. 
2. In XML, a node in document. Each element starts with a and ends with .}
\glossitem{federation}{The dynamic combination of information from separate sources of information.}
\glossitem{FITS}{(Flexible Image Transport System) The IAU-approved standard format for astronomical data.}
\glossitem{FTP}{(File Transfer Protocol) A protocol used to exchange files over the internet.}
\glossitem{footprint}{The region of the sky that has been observed by one or more telescopes.}
\glossitem{GPL}{(GNU Public License) A software license which allows for redistribution but requires both original and modified source code to be made available.}
\glossitem{Grid}{Massive distributed computing capabilities currently available on the Internet.}
\glossitem{grid computing}{Applying the resources of many computers in a network to a single problem at the same time.}
\glossitem{GUI}{(Graphical User Interface) A graphics-based user interface that incorporates movable windows, icons and a mouse.}
\glossitem{GWS}{(Grid and Web Services)}
\glossitem{HPC}{(High Performance Computing) Typically refers to supercomputers used in scientific research.}
\glossitem{HTML}{(Hypertext Markup Language) A standard document format used on most web pages which makes it easy for one document to refer to another.}
\glossitem{HTTP}{Communications protocol used to access most web pages.}
\glossitem{HTTPS}{A secure version of the HTTP protocol.}
\glossitem{HTTP GET}{An HTTP request where any parameters of the request are included in the URL itself.}
\glossitem{HTTP POST}{An HTTP request where data is sent to the server.}
\glossitem{IDL}{(Interactive Data Language) A popular data analysis programming language used by scientists.}
\glossitem{instance document}{An XML document that conforms to a schema.}
\glossitem{interface}{A structured interaction between two entities, often a client and a server.}
\glossitem{interoperability}{The ability of software and/or hardware on different machines from different vendors or sources to share data or collaborate without special effort on the part of the user.}
\glossitem{interpreted language}{A programming language that is parsed and executed without needing any explicit compilation.}
\glossitem{IRAF}{(Image Reduction and Analysis Facility) A software system for astronomical data analysis including both tools, libraries and languages developed primarily at the National Optica Astronomy Observatories (NOAO).}
\glossitem{IVOA}{(International Virtual Observatory Alliance) An international collaboration formed in June 2002 to coordinate Virtual Observatory activities worldwide.}
\glossitem{IVORN, IVOA identifier}{A standardized, unique ID used in a number of contexts within the VO.}
\glossitem{Java}{An object-oriented, platform-independent programming language, developed by Sun, and modeled after C++.}
\glossitem{JDBC}{(Java Database Connectivity) The standard Java interface for access to SQL -based DBMS ’s.}
\glossitem{KDIG}{(Knowledge Discovery Interst Group)}
\glossitem{markup language}{A language for annotating a document to enable each component to be appropriately formatted, displayed, or used.}
\glossitem{metadata}{Information or labels that describe data.}
\glossitem{middleware}{An intermediate level of computer software typically used to provide a common interface to heterogeneous lower level components.}
\glossitem{MIME}{(Multipurpose Internet Mail Extensions) The most common protocol for encoding, transmitting, and decoding non-text files via e-mail.}
\glossitem{Mirage}{VO-enabled tool for exploratory analysis and visualization of images and multi-dimensional numerical data.}
\glossitem{mosaic, mosaicking}{A virtual observation made by combining multiple observations at different positions resulting in a combination of images that abut and/or overlap into a single larger image.}
\glossitem{MySQL}{An open source SQL relational database management system ( RDBMS).}
\glossitem{name resolver}{A service that translates object names into astronomical coordinates.}
\glossitem{namespace}{1. The set of names in a naming system.
2. In XML, a collection of names, identified by a URI reference, that are used in XML documents as element types and attribute names.}
\glossitem{NED}{(NASA Extragalactic Database) An astronomical resource containing extensive information on extragalactic objects. NED also provides a name resolver.}
\glossitem{NESSSI}{(NVO Extensible Scalable Secure Service Infrastructure) A VO web service that runs secure, asynchronous services on the Grid.}
\glossitem{OAI}{(Open Archive Initiative) A protocol that defines an interface for the sharing of metadata.}
\glossitem{OASIS}{(On-line Archive Science Information Services) A VO Tool to access and display astronomical image and catalog data.}
\glossitem{object-oriented}{Programming based on the concept of an “object”, a data structure associated with specific routines that define the behavior of the object.}
\glossitem{ontology}{A formal description of the vocabulary used in a field, especially describing the relationships between various concepts within that subject.}
\glossitem{open source}{Software, usually free, created by a development community where the source code is distributed as well as compiled code.}
\glossitem{OpenSkyQuery}{A VO Service that enables crossmatching of astronomical catalogs and selection of catalog subsets.}
\glossitem{operating system}{The program framework in which all other programs run on a computer, e.g. Windows XP, MacOS? X, Linux, etc.}
\glossitem{parameters}{Inputs to or elements in a system which can be varied.}
\glossitem{parse}{To separate into more easily understood parts.}
\glossitem{peer-to-peer}{Communication between two or more computers where the protocol is symmetric between the participants so that each participant can make the same requests or give the same responses. This contrasts with client-server protocols}
\glossitem{Perl}{A programming language frequently used for web scripts and to process data passed via HTML forms.}
\glossitem{PHP}{(PHP [personal home page] Hypertext Preprocessor) A scripting language used to create dynamic Web pages.}
\glossitem{PLASTIC}{(PLatform for AStronomical Tool InterConnection? ) A protocol that allows collaboration between multiple processes running on the user’s desktop.}
\glossitem{platform}{The combination of a computer’s operating system software and hardware.}
\glossitem{portal}{A web site that serves as a starting point to other destinations or services on the web.}
\glossitem{protocol}{A set of rules that define the interactions between two or more components.}
\glossitem{proxy}{A piece of software that acts on behalf of a user or another piece of software. An agent is a client proxy.}
\glossitem{proxy certificate}{A certificate that is used in the place of another, typically with a limited lifetime.}
\glossitem{query}{To interrogate a collection of data such as records in a database.}
\glossitem{RDBMS}{(Relational DataBase? Management System) A DBMS that uses represents data using a relational database.}
\glossitem{RDF}{(Resource Description Framework) A recommendation from the W3C for creating metadata structures that define data on the web.}
\glossitem{registry}{The “yellow pages” for the VO. Collects and stores basic information about archives, data collections, databases, and other resources.}
\glossitem{relational database}{A database that stores data in a structure consisting of one or more tables (aka relations) of rows and columns, which may be interconnected.}
\glossitem{resource metadata}{The metadata that describes services and data resources available in the VO.}
\glossitem{REST}{(Representational State Transfer) An approach to web services that uses the standard HTTP GET and POST protocols.}
\glossitem{RMI}{(Remote Method Invocation) A Java protocol for distributed computing.}
\glossitem{ROME}{(Request-Object Management Environment) A VO tool to manage the execution of a task that requires many subtasks.}
\glossitem{RPC}{(Remote Procedure Call) Protocols for distributed computing where the interaction is represented as the client computer invoking discrete services/calls on the server.}
\glossitem{RSS}{(Rich Site Summary). An XML format for sharing content among different Web sites such as news items.}
\glossitem{Ruby}{An object-oriented programming language.}
\glossitem{SAX}{(Simple API for XML) A standardized interface for parsing XML documents using callbacks.}
\glossitem{schema}{1. A description of the structure and rules an XML document must satisfy. 
2. In SQL, a description of the tables and columns in the database.}
\glossitem{script}{A simple program usually written in an interpreted language.}
\glossitem{semantics}{The expression of the meaning of symbols or names. In the VO, the actual scientific meaning of data and services.}
\glossitem{serialization}{The process of converting an object into a format that can be stored or transmitted across a network.}
\glossitem{server}{A computer system in a network whose services may be invoked by one or more clients.}
\glossitem{servlet}{A program, typically Java, that runs on a web server in response to a web request.}
\glossitem{SESAME}{A web service interface to the SIMBAD name resolver.}
\glossitem{sexagesimal}{Numeral system with number 60 as the base.}
\glossitem{SExtractor}{(Source Extractor) A tool that detects sources in astronomical images.}
\glossitem{SIA, SIAP}{(Simple Image Access Protocol) A VO protocol thatsupports queries for images available in a given data collection near a given position on the sky.}
\glossitem{SIMBAD}{(Set of Identifications, Measurements and Bibliography for Astronomical Data) An astronomical database provides extensive information on both galactic and extragalactic objects. SIMBAD also provides a name resolver.}
\glossitem{Simple Spectral Access}{See SSAP.}
\glossitem{SkyNode}{A VO protocol (and the services that implement it) that provides an ADQL interface to astronomical databases.}
\glossitem{SkyPortal}{A web site that supports translation of a user ADQL query into queries of one or more SkyNodes.}
\glossitem{SkyServer}{Web service that presents data from the Sloan Digital Sky Survey.}
\glossitem{SkyView}{Web site and VO-enabled distributable tool that generates images from survey data.}
\glossitem{SMTP}{(Simple Mail Transfer Protocol) A protocol used to send and receive email.}
\glossitem{SOA}{(Service Oriented Architecture) An approach to distributed computing that focuses on services that communicate with each other.}
\glossitem{SOAP}{(Simple Object Access Protocol) A protocol for invoking remote services by exchanging XML -based messages.}
\glossitem{socket}{The low-level software element that makes a connection to the network. Normally a client connects to a socket on a server.}
\glossitem{source code}{The version of a program normally written or edited by a programmer and either compiled into an executable program, or run directly using an interpreter (see interpreted languag).}
\glossitem{SQL}{(Structured Query Language) The standard language used to communicate with RDBMS s}
\glossitem{SQL Server}{Microsoft’s RDBMS software.}
\glossitem{SRB}{(Storage Resource Broker) Middleware developed at the San Diego Supercomputing Center that provides standardized access to a number of very large data resources.}
\glossitem{SSAP}{(Simple Spectral Access Protocol) A protocol that returns a set of spectra in a specified region of the sky. Similar to SIA but has many more options.}
\glossitem{SSDL}{(SOAP Service Description Language) SSDL is a SOAP -centric description language for web services that enables protocol -based integration.}
\glossitem{SSL}{(Secure Sockets Layer) A protocol for managing the security of a message transmission over the Internet.}
\glossitem{standalone application}{A computer program capable of operating without external resources.}
\glossitem{STC}{(Space-Time Coordinates) An IVOA standard for describing a region or position in both space and time.}
\glossitem{STILTS}{(STIL Tool Set) A set of VO tools for processing of tabular data based on the UK Starlink Tables Infrastructure.}
\glossitem{TLA}{(Three Letter Acronym) A tribute to the use of acronyms in the computer field.}
\glossitem{token}{An item in a string of text that can be separated out by a parser, such as a single word in a sentence or a number in a comma-delimited list.}
\glossitem{TOMCAT}{An HTTP server that can run Java servlets.}
\glossitem{TOPCAT}{(Tool for OPerations on Catalogs And Tables) An interactive graphical viewer and editor for tabular data, designed for but not limited to astronomical tables.}
\glossitem{Treeview}{A VO-enabled viewer for hierarchical file structures.}
\glossitem{UCD}{(Unified Content Descriptors) A formal vocabulary for astronomical data that is controlled by the IVOA.}
\glossitem{URI}{(Uniform Resource Identifier) An address standard for a resource available on the Internet.}
\glossitem{URL}{(Uniform Resource Locator) The global address of documents and other resources on the World Wide Web. The first part of the address specifies the protocol to be used when accessing the resource, the remainder describes its network location.}
\glossitem{validator}{A tool that checks some element of a system for conformance to a standard.}
\glossitem{virtual data}{Data product that is dynamically generated when needed.}
\glossitem{VOClient}{A software suite callable from many languages which implements data access in the VO.}
\glossitem{VOEvent}{A VO standard for representing, transmitting, publishing and archiving the discovery of a transient celestial event.}
\glossitem{VOEventNet}{A peer-to-peer cyberinfrastructure to enable rapid and federated observations of the dynamic night sky.}
\glossitem{VOPlot}{A VO Tool for visualizing astronomical data from VOTable sources.}
\glossitem{VOSpace}{A distributed storage concept for the VO.}
\glossitem{VOStat}{A VO and web-enabled statistics package.}
\glossitem{VOTable}{An XML -based encoding scheme for astronomical tables and catalogs, established by the IVOA in order to provide an unambiguous way to transmit tables between computer programs}
\glossitem{W3C}{(World Wide Web Consortium) An international consortium where member organizations, a full-time staff, and the public work together to develop web standards.}
\glossitem{WCS}{(World Coordinate System) A detailed specification of the conversion between coordinates within a file and physical coordinates, especially between pixel and celestial coordinates in an image.}
\glossitem{WCS Fixer}{A VO web service that corrects the WCS information in a given FITS image.}
\glossitem{web service}{Software available over the web using a standardized XML messaging system.}
\glossitem{WESIX}{A VO web service to the standard astronomical image analysis package SExtractor together with a crossmatching service.}
\glossitem{wget}{Free software package for retrieving files using HTTP, HTTPS and FTP.}
\glossitem{workflow}{A sequence or network of tasks and associated information needed to pass from one task to another to accomplish some goal.}
\glossitem{WS}{see web service}
\glossitem{WSDL}{(Web Services Description Language) An XML document that describes and locates a web service.}
\glossitem{XMatch}{see crossmatch}
\glossitem{XML}{(eXtensible Markup Language) A markup language that provides a file format for representing data.}
\glossitem{XML-RPC}{(XML-Remote Procedure Call) A web service protocol that utilizes XML technology to implement an RPC protocol.}
\glossitem{XOP}{(XML Binary Optimized Packaging) A standard specifying how binary data should be represented in XML.}
\glossitem{XPath}{(XML Path Language) a language that describes how to locate specific content within an XML document.}
\glossitem{XQuery}{A standard language for querying XML data.}
\glossitem{XSL}{(eXtensible Stylesheet Language) A standard for describing how to transform XML documents into other documents in either XML or other formats.}
\glossitem{XSLT}{(eXtensible Stylesheet Language Transformations) A conversion tool that implements XSL.}
\glossitem{YAML}{(YAML Ain’t Markup Language) A data serialization
language based on XML and other languages.}
